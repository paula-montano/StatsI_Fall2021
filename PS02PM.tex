% Options for packages loaded elsewhere
\PassOptionsToPackage{unicode}{hyperref}
\PassOptionsToPackage{hyphens}{url}
%
\documentclass[
]{article}
\usepackage{amsmath,amssymb}
\usepackage{lmodern}
\usepackage{ifxetex,ifluatex}
\ifnum 0\ifxetex 1\fi\ifluatex 1\fi=0 % if pdftex
  \usepackage[T1]{fontenc}
  \usepackage[utf8]{inputenc}
  \usepackage{textcomp} % provide euro and other symbols
\else % if luatex or xetex
  \usepackage{unicode-math}
  \defaultfontfeatures{Scale=MatchLowercase}
  \defaultfontfeatures[\rmfamily]{Ligatures=TeX,Scale=1}
\fi
% Use upquote if available, for straight quotes in verbatim environments
\IfFileExists{upquote.sty}{\usepackage{upquote}}{}
\IfFileExists{microtype.sty}{% use microtype if available
  \usepackage[]{microtype}
  \UseMicrotypeSet[protrusion]{basicmath} % disable protrusion for tt fonts
}{}
\makeatletter
\@ifundefined{KOMAClassName}{% if non-KOMA class
  \IfFileExists{parskip.sty}{%
    \usepackage{parskip}
  }{% else
    \setlength{\parindent}{0pt}
    \setlength{\parskip}{6pt plus 2pt minus 1pt}}
}{% if KOMA class
  \KOMAoptions{parskip=half}}
\makeatother
\usepackage{xcolor}
\IfFileExists{xurl.sty}{\usepackage{xurl}}{} % add URL line breaks if available
\IfFileExists{bookmark.sty}{\usepackage{bookmark}}{\usepackage{hyperref}}
\hypersetup{
  pdftitle={PS02\_PM},
  pdfauthor={Paula},
  hidelinks,
  pdfcreator={LaTeX via pandoc}}
\urlstyle{same} % disable monospaced font for URLs
\usepackage[margin=1in]{geometry}
\usepackage{color}
\usepackage{fancyvrb}
\newcommand{\VerbBar}{|}
\newcommand{\VERB}{\Verb[commandchars=\\\{\}]}
\DefineVerbatimEnvironment{Highlighting}{Verbatim}{commandchars=\\\{\}}
% Add ',fontsize=\small' for more characters per line
\usepackage{framed}
\definecolor{shadecolor}{RGB}{248,248,248}
\newenvironment{Shaded}{\begin{snugshade}}{\end{snugshade}}
\newcommand{\AlertTok}[1]{\textcolor[rgb]{0.94,0.16,0.16}{#1}}
\newcommand{\AnnotationTok}[1]{\textcolor[rgb]{0.56,0.35,0.01}{\textbf{\textit{#1}}}}
\newcommand{\AttributeTok}[1]{\textcolor[rgb]{0.77,0.63,0.00}{#1}}
\newcommand{\BaseNTok}[1]{\textcolor[rgb]{0.00,0.00,0.81}{#1}}
\newcommand{\BuiltInTok}[1]{#1}
\newcommand{\CharTok}[1]{\textcolor[rgb]{0.31,0.60,0.02}{#1}}
\newcommand{\CommentTok}[1]{\textcolor[rgb]{0.56,0.35,0.01}{\textit{#1}}}
\newcommand{\CommentVarTok}[1]{\textcolor[rgb]{0.56,0.35,0.01}{\textbf{\textit{#1}}}}
\newcommand{\ConstantTok}[1]{\textcolor[rgb]{0.00,0.00,0.00}{#1}}
\newcommand{\ControlFlowTok}[1]{\textcolor[rgb]{0.13,0.29,0.53}{\textbf{#1}}}
\newcommand{\DataTypeTok}[1]{\textcolor[rgb]{0.13,0.29,0.53}{#1}}
\newcommand{\DecValTok}[1]{\textcolor[rgb]{0.00,0.00,0.81}{#1}}
\newcommand{\DocumentationTok}[1]{\textcolor[rgb]{0.56,0.35,0.01}{\textbf{\textit{#1}}}}
\newcommand{\ErrorTok}[1]{\textcolor[rgb]{0.64,0.00,0.00}{\textbf{#1}}}
\newcommand{\ExtensionTok}[1]{#1}
\newcommand{\FloatTok}[1]{\textcolor[rgb]{0.00,0.00,0.81}{#1}}
\newcommand{\FunctionTok}[1]{\textcolor[rgb]{0.00,0.00,0.00}{#1}}
\newcommand{\ImportTok}[1]{#1}
\newcommand{\InformationTok}[1]{\textcolor[rgb]{0.56,0.35,0.01}{\textbf{\textit{#1}}}}
\newcommand{\KeywordTok}[1]{\textcolor[rgb]{0.13,0.29,0.53}{\textbf{#1}}}
\newcommand{\NormalTok}[1]{#1}
\newcommand{\OperatorTok}[1]{\textcolor[rgb]{0.81,0.36,0.00}{\textbf{#1}}}
\newcommand{\OtherTok}[1]{\textcolor[rgb]{0.56,0.35,0.01}{#1}}
\newcommand{\PreprocessorTok}[1]{\textcolor[rgb]{0.56,0.35,0.01}{\textit{#1}}}
\newcommand{\RegionMarkerTok}[1]{#1}
\newcommand{\SpecialCharTok}[1]{\textcolor[rgb]{0.00,0.00,0.00}{#1}}
\newcommand{\SpecialStringTok}[1]{\textcolor[rgb]{0.31,0.60,0.02}{#1}}
\newcommand{\StringTok}[1]{\textcolor[rgb]{0.31,0.60,0.02}{#1}}
\newcommand{\VariableTok}[1]{\textcolor[rgb]{0.00,0.00,0.00}{#1}}
\newcommand{\VerbatimStringTok}[1]{\textcolor[rgb]{0.31,0.60,0.02}{#1}}
\newcommand{\WarningTok}[1]{\textcolor[rgb]{0.56,0.35,0.01}{\textbf{\textit{#1}}}}
\usepackage{graphicx}
\makeatletter
\def\maxwidth{\ifdim\Gin@nat@width>\linewidth\linewidth\else\Gin@nat@width\fi}
\def\maxheight{\ifdim\Gin@nat@height>\textheight\textheight\else\Gin@nat@height\fi}
\makeatother
% Scale images if necessary, so that they will not overflow the page
% margins by default, and it is still possible to overwrite the defaults
% using explicit options in \includegraphics[width, height, ...]{}
\setkeys{Gin}{width=\maxwidth,height=\maxheight,keepaspectratio}
% Set default figure placement to htbp
\makeatletter
\def\fps@figure{htbp}
\makeatother
\setlength{\emergencystretch}{3em} % prevent overfull lines
\providecommand{\tightlist}{%
  \setlength{\itemsep}{0pt}\setlength{\parskip}{0pt}}
\setcounter{secnumdepth}{-\maxdimen} % remove section numbering
\ifluatex
  \usepackage{selnolig}  % disable illegal ligatures
\fi

\title{PS02\_PM}
\author{Paula}
\date{10/5/2021}

\begin{document}
\maketitle

\hypertarget{r-markdown}{%
\subsection{R Markdown}\label{r-markdown}}

This is an R Markdown document. Markdown is a simple formatting syntax
for authoring HTML, PDF, and MS Word documents. For more details on
using R Markdown see \url{http://rmarkdown.rstudio.com}.

When you click the \textbf{Knit} button a document will be generated
that includes both content as well as the output of any embedded R code
chunks within the document. You can embed an R code chunk like this:

\begin{Shaded}
\begin{Highlighting}[]
\FunctionTok{summary}\NormalTok{(cars)}
\end{Highlighting}
\end{Shaded}

\begin{verbatim}
##      speed           dist       
##  Min.   : 4.0   Min.   :  2.00  
##  1st Qu.:12.0   1st Qu.: 26.00  
##  Median :15.0   Median : 36.00  
##  Mean   :15.4   Mean   : 42.98  
##  3rd Qu.:19.0   3rd Qu.: 56.00  
##  Max.   :25.0   Max.   :120.00
\end{verbatim}

\hypertarget{including-plots}{%
\subsection{Including Plots}\label{including-plots}}

You can also embed plots, for example:

\includegraphics{PS02PM_files/figure-latex/pressure-1.pdf}

Note that the \texttt{echo\ =\ FALSE} parameter was added to the code
chunk to prevent printing of the R code that generated the plot.

\hypertarget{load-libraries}{%
\section{load libraries}\label{load-libraries}}

\hypertarget{set-wd}{%
\section{set wd}\label{set-wd}}

\hypertarget{clear-global-.envir}{%
\section{clear global .envir}\label{clear-global-.envir}}

\hypertarget{remove-objects}{%
\section{remove objects}\label{remove-objects}}

rm(list=ls())

\hypertarget{detach-all-libraries}{%
\section{detach all libraries}\label{detach-all-libraries}}

detachAllPackages \textless- function() \{ basic.packages \textless-
c(``package:stats'', ``package:graphics'', ``package:grDevices'',
``package:utils'', ``package:datasets'', ``package:methods'',
``package:base'') package.list \textless-
search(){[}ifelse(unlist(gregexpr(``package:'', search()))==1, TRUE,
FALSE){]} package.list \textless- setdiff(package.list, basic.packages)
if (length(package.list)\textgreater0) for (package in package.list)
detach(package, character.only=TRUE) \} detachAllPackages()

\hypertarget{load-libraries-1}{%
\section{load libraries}\label{load-libraries-1}}

pkgTest \textless- function(pkg)\{ new.pkg \textless- pkg{[}!(pkg \%in\%
installed.packages(){[}, ``Package''{]}){]} if (length(new.pkg))
install.packages(new.pkg, dependencies = TRUE) sapply(pkg, require,
character.only = TRUE) \}

library(stringr) library(dplyr) library(tidyverse) library(ggplot2)
library(viridisLite)

\hypertarget{load-necessary-packages}{%
\section{Load necessary packages}\label{load-necessary-packages}}

lapply(c(``stringr''), pkgTest) lapply(c(``dplyr''), pkgTest)
lapply(c(``tidyverse''), pkgTest) lapply(c(``ggplot2''), pkgTest)

\hypertarget{set-working-directory}{%
\section{set working directory}\label{set-working-directory}}

setwd(``\textasciitilde/Documents/GitHub/StatsI\_Fall2021/StatsI\_PM/problemSets/Completed
problemSets /PS02\_PM'')

\hypertarget{problem-1}{%
\section{Problem 1}\label{problem-1}}

\#\#H0: The variables are statistically independent \#\#Ha: The
variables are statistically dependent

\#\#a) Calculate a test-statistic (χ2 statistic)

\#\#f observed = fo = observed frequency = the raw count \#\#f expected
= fe = what we would expect for independent samples \#\#= Row total /
Grand total*Column total \#\#If H0 is true, then we would expect f
observed = f expected

Answer: X2 = 3.79

\#\#b) Calculate p-value from test-statistic (χ2 statistic)\\
\#\#df = (rows-1) (columns-1) \#\#df = (3-1) (2-1) \#\#p-value
pchisq(3.79, df=2, lower.tail=FALSE)

\#\#p-value = 0.1503183 \#\# If p ≤ α we conclude that the evidence
supports the alternative hypothesis Ha. \#\# If p \textgreater{} α we
cannot reject the null hypothesis H0.

\#\#What we conclude from α = .1 \#\# Our p-value is greater than α,
therefore we cannot reject our null hypothesis. The variables are not
statistically dependent.

\#\#Complete question 1 was done by hand.

\hypertarget{problem-2}{%
\section{Problem 2}\label{problem-2}}

westBengal \textless-
read.csv(``\url{https://raw.githubusercontent.com/kosukeimai/qss/master/PREDICTION/women.csv}'')

\hypertarget{analysing-the-dataset}{%
\section{Analysing the dataset}\label{analysing-the-dataset}}

str(westBengal) head(westBengal) summary(westBengal)

\#\#State a null and alternative (two-tailed) hypothesis

\#\#*(a) \#\#Ho: When GP was reserved for women leaders the number of
new or repaired drinking-water facilities decreased in the village.
\#\#Ha: When GP was reserved for women leaders the number of new or
repaired drinking-water facilities increased in the village.

\#\#(b)

\hypertarget{problem-3}{%
\section{Problem 3}\label{problem-3}}

fruitfly
\textless-read.csv(``\url{http://stat2.org/datasets/FruitFlies.csv}'')

\hypertarget{analysing-the-dataset-1}{%
\section{Analysing the dataset}\label{analysing-the-dataset-1}}

\#\#Import the data set and obtain summary statistics and examine the
distribution of the overall lifespan of the fritflies.

str(fruitfly) head(fruitfly) summary(fruitfly)

lifespan\_histogram \textless- ggplot(fruitfly, aes(x = Longevity)) +
geom\_histogram(bins = 10, color = ``blue'', fill = ``lightblue'') +
labs(x = ``Longevity'', y = ``Frequency'', title = ``Histogram of
lifespan of the fruitflies'') + theme\_minimal()

\#\#Plot lifespan vs thorax.

Plot\_lifespan\_vs\_thorax \textless-
plot(fruitfly\(Longevity, fruitfly\)Thorax, main = ``Scatter Plot of
lifespan vs thorax'', xlab = ``Predictor Longevity on X axis'', ylab =
``Target Thorax on y axis'')

\#\#(2)

lifespan\_thorax \textless- lm(Longevity \textasciitilde{} Thorax, data
= fruitfly) summary(lifespan\_thorax) class(lifespan\_thorax)

ggplot(aes(Thorax, Longevity), data = fruitfly) + geom\_point() +
geom\_smooth(method = ``lm'', formula = y \textasciitilde{} x)
str(lifespan\_thorax)

\#\#The variables lifespan and thorax have a positive relation. The plot
depicts a positive linear relationship when a value in X increases it
also increses in Y. The correlation coefficient shows a strong
association between liespan and thorax variable.

\#\#(3) Plot\_lifespan\_thorax \textless- ggplot(aes(Longevity, Thorax),
data = fruitfly) + geom\_point(alpha = 0.4) + labs(x = ``Lifespan'', y =
``Thorax'', title = ``Scatter Plot of lifespan vs thorax'')

plot(fruitfly\(Longevity, fruitfly\)Thorax, main = ``Scatter Plot of
lifespan on thorax'', xlab = ``Predictor Longevity on X axis'', ylab =
``Target Thorax on y axis'') lm(fruitfly\(Thorax ~ fruitfly\)Longevity)
abline(lm(fruitfly\(Thorax ~ fruitfly\)Longevity), col = ``red'')

\#\#The red slope shows the positive relation between lifespan and
thorax in the fruitflies. The slope is steep and shows the strong
association between the input and output variables.

\#\#(4)

lm(Longevity\textasciitilde Thorax, data = fruitfly)

\#\#(5) z90 \textless- qnorm((1-.90) / 2, lower.tail = FALSE) n
\textless-
length(na.omit(fruitfly\(lifespan_thorax)) fruitfly_mean <- mean(fruitfly\)lifespan\_thorax,
na.rm = TRUE) fruitfly\_sd \textless- sd(fruitfly\$lifespan\_thorax,
na.rm = TRUE) lower\_90 \textless- fruitfly\_mean - (z90 * (fruitfly\_sd
/ sqrt(n))) upper\_90 \textless- fruitfly\_mean + (z90 * (fruitfly\_sd /
sqrt(n))) confint90 \textless- c(lower\_90, upper\_90)

\#\#Function confint() confint(lifespan\_thorax, parm = 0.90, level =
0.90)

\end{document}
